\documentclass[a4paper,twoside,10pt]{article}
\usepackage{a4wide,graphicx,fancyhdr,clrscode,wrapfig,tabularx,amsmath,amssymb,enumerate}
\usepackage[usenames, dvipsnames]{color}
\usepackage[pdftex,colorlinks,linkcolor=Black]{hyperref}

\begin{document}
\thispagestyle{empty}
\begin{codebox}
\Procname{$\proc{Intersection}(A,B,C,D)$}
\li $E \gets B-A$
\li $F \gets D-C$\\
    \Comment Make a vector perpendicular to vector E.
\li $P.x \gets -E.y$
\li $P.y \gets E.x$
\li $numerator \gets ((A-C) \bullet P)$ \Comment Calculate dot-product.
\li $denominator \gets (F \bullet P)$ \Comment Calculate dot-product.
\li \If     $denominator <> 0$
\li \Then   $h \gets numerator / denominator$
\li         \If     $0 < h < 1$
\li         \Then   $result := True$ \Comment There is an intersection.
\li         \Else   $result := False$ \Comment There is no intersection.
            \End
\li \Else   $result := False$ \Comment There is no intersection.
    \End
\end{codebox}

Figure 2: pseudo-code of the checking for intersections data structure.
\end{document}
