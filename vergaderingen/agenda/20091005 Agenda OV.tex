%
%  untitled
%
%  Created by Etienne van Delden on 2009-03-31.
%  Copyright (c) 2009 __MyCompanyName__. All rights reserved.
%
\documentclass[a4paper]{article}

% Use utf-8 encoding for foreign characters
\usepackage[utf8]{inputenc}
\usepackage[dutch]{babel}

% Setup for fullpage use
%\usepackage{fullpage}

% Uncomment some of the following if you use the features
%
% Running Headers and footers
%\usepackage{fancyhdr}
\usepackage{hyperref}


% Multipart figures
%\usepackage{subfigure}

% More symbols
\usepackage{amsmath}
\usepackage{amssymb}
%\usepackage{latexsym}

% Surround parts of graphics with box
\usepackage{boxedminipage}

% Package for including code in the document
\usepackage{listings}

% If you want to generate a toc for each chapter (use with book)
%\usepackage{minitoc}

% This is now the recommended way for checking for PDFLaTeX:
\usepackage{ifpdf}

%\newif\ifpdf
%\ifx\pdfoutput\undefined
%\pdffalse % we are not running PDFLaTeX
%\else
%\pdfoutput=1 % we are running PDFLaTeX
%\pdftrue
%\fi

\ifpdf
\usepackage[pdftex]{graphicx}
\else
\usepackage{graphicx}
\fi
\title{Agenda OGO 2.1 Vergadering 2009-2010}
\date{}


\begin{document}

\ifpdf
\DeclareGraphicsExtensions{.pdf, .jpg, .tif}
\else
\DeclareGraphicsExtensions{.eps, .jpg}
\fi

\maketitle

\section{\underline{Opening Vergadering}} % (fold)
\label{sec:opening_vergadering}
\begin{tabular}{ll}
  Datum:      & 05-10-2009\\
  Begin tijd: & 14:15\\
  Voorzitter: & Etienne van Delden\\
  Notulist:   & Jan\^ot Sijen\\
  \multicolumn{2}{l}{Aanwezigheid leden}
\end{tabular}


% section opening_vergadering (end)

\section{\underline{Vaststelling van de Agenda}} % (fold)
\label{sec:vaststelling_van_de_agenda}

% section vaststelling_van_de_agenda (end)

%\section{\underline{Verslag vorige vergadering}} % (fold)
%\label{sec:bespreking_van_de_vorige_notulen}
%
%% section bespreking_van_de_vorige_notulen (end)

\section{\underline{Mededelingen}} % (fold)
\label{sec:mededelingen}
% section mededelingen (end)

\section{\underline{Correspondentie}} % (fold)
\label{sec:post_in_uit}

% section post_in_uit (end)

\section{\underline{Update Algorithms}} % (fold)
\label{sec:feedback_werkplan}

% section feedback_werkplan (end)

\section{\underline{Voortgang OGO}} % (fold)
\label{sec:voortgang_ogo}
    \begin{itemize}
        \item Vraag: Moet je tegen de klok in alle punten langs gaan of steeds de lexicografische kleinste? (De documentatie en het voorbeeld erin spreken elkaar tegen)
    \end{itemize}
% section voortgang_ogo (end)

\section{\underline{Planning komende week}} % (fold)
\label{sec:planning_komende_week}

% section planning_komende_week (end)

\section{\underline{Rondvraag}} % (fold)
\label{sec:rondvraag}

% section rondvraag (end)

\section{\underline{Punten voor de volgende vergadering}} % (fold)
\label{sec:punten_voor_de_volgende_vergadering}

% section punten_voor_de_volgende_vergadering (end)

\section{\underline{Sluiting}} % (fold)
\label{sec:sluiting}
\small{\emph{Logboek invullen!}}
% section sluiting (end)

\section{Bijlage: Deadlines} % (fold)
\label{sec:bijlage_deadlines}

  \begin{tabular}{|p{5cm}|p{3cm}|p{2.5cm}|p{4cm}|}
      \hline
      \textbf{Subject} & \textbf{Soft Deadline} & \textbf{Hard deadline} & \textbf{Place}\\
      \hline
      \hline
      Plan of action & 16 September & 18 september & Tutor\\
      \hline
      Closed Algorithm & 23 September& & 10.46\\
      \hline
      Open Algorithm & 7 October & & 10.46\\
      \hline
      Intersection Algorithm & 7 October& & 10.46\\
      \hline
      2-to-5 Algorithm & 14 October& & 10.46\\
      \hline
      Conceptual report & 19 October & 9 November& Tutor / K. Buchin (at HG 7.22) \\
      \hline
      Code (soft deadline) & 23 November & 7 December & 10.46\\
      \hline
      Final report (soft deadline) & 2 December & 11 December & 10.46 / K. Buchin (at HG 7.22)\\
      \hline
      Presentation & 9 December & &10.46\\
      \hline
      Final meeting & 14 December & & AUD 10\\
      \hline
  \end{tabular} \\ \\

% section bijlage_deadlines (end)

\end{document}
