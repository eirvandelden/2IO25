%
%  Notulen 16-11-2009
%  OGO 2.1 Groep 3
%
\documentclass[a4paper]{article}

% Packages!
\usepackage[utf8]{inputenc}
\usepackage[dutch]{babel}
\usepackage{hyperref}
\usepackage{amsmath}
\usepackage{amssymb}
\usepackage{boxedminipage}
\usepackage{listings}
\usepackage{ifpdf}

\ifpdf
\usepackage[pdftex]{graphicx}
\else
\usepackage{graphicx}
\fi

\title{Notulen Vergadering OGO 2.1 Groep 3 2009-2010}
\date{}

\begin{document}

\ifpdf
\DeclareGraphicsExtensions{.pdf, .jpg, .tif}
\else
\DeclareGraphicsExtensions{.eps, .jpg}
\fi

\section{\underline{Opening Vergadering}} % (fold)
\label{sec:opening_vergadering}
\begin{tabular}{ll}
  Datum:      & 16-11-2009\\
  Begin tijd: & 13:45\\
  Voorzitter: & Etienne van Delden\\
  Notulist:   & Jan\^ot Sijen\\
  & \\
  Aanwezig:   & Etienne van Delden\\
              & Tim Hermans\\
              & Jan\^ot Sijen\\
              & Robin Wolffensperger\\
              & Ron Vanderfeesten (tutor)\\
  Afwezig:    & Tom van der Hoek \\
\end{tabular}

% section opening_vergadering (end)
\section{\underline{Vaststelling Agenda:}} % (fold)
\label{sec:vaststellingagenda}

\begin{itemize}
\item Geen aanvullingen.
\end{itemize}

% section vaststellingagenda (end)

\section{\underline{Notulen vorige keer:}} % (fold)
\label{sec:notulenvorigekeer}

\begin{itemize}
\item Ron heeft niet gekeken, sorry!
\item Geen opmerkingen.
\end{itemize}

% section notulenvorigekeer (end)

\section{\underline{Mededelingen:}} % (fold)
\label{sec:mededelingen}

\begin{itemize}
\item Geen conceptverslagen nagekeken. Het conceptverslag deze week wordt gewoon met de andere nagekeken.
\end{itemize}

% section mededelingen (end)

\section{\underline{Concepverslag:}} % (fold)
\label{sec:concepverslag}

\begin{itemize}
\item Meer tekst, minder code, plaatjes. Perfect!
\item HTML pagina is niet een echte referentie...
\item Line/line intersection staat misschien ook in het Lineaire Algebra boek; dus we kunnen dat refereren.
\item Het wordt nog gelezen door Ron, maar het ziet er wel al goed uit!
\item Bij de paper referentie: het jaar erbij zetten
\item Aan het einde van de dag een versie van het conceptverslag inleveren.
\end{itemize}

% section concepverslag (end)

\section{\underline{Voortgang OGO:}} % (fold)
\label{sec:voortgangogo}

\begin{itemize}
\item Vorige week was toch nog de for-loop!
\item Backtracking werkt, maar het is heel langzaam.
\item Misschien omschrijven naar iteratief? (Jumpen, niet callen!)
\item Maar eerst: kijken of we backtracking effecient kunnen maken.
\item Backtracking met een range limitation: Dan hebben we Greedy algoritme!
\item Backtracking is basically brute forcing. Er zijn optimalizaties; zoals afstanden tussen punten opslaan. (Of bijvoorbeeld pruning.)
\item Alternatieven: Greedy and Randomize
\item Greedy: Ik sta ergens, en ik kies één mogelijkheid waarvan ik denk dat die het beste is. (Zoiets als de for-loop?)
\item Randomize kan alleen in selecte gevallen. In deze situatie is het misschien iets minder handig.
\item For-loop; kunnen we dit niet oplossen door het begin- en eindpunt slim te kiezen?
\end{itemize}

% section voortgangogo (end)

\section{\underline{Actielijst:}} % (fold)
\label{sec:actielijst}

\begin{itemize}
\item Verslag afmaken
\item Algoritme fixen. (Ofwel Greedy iteratief maken, of de for-loop verbeteren.)
\end{itemize}

% section actielijst (end)

\section{\underline{Rondvraag:}} % (fold)
\label{sec:rondvraag}

\begin{itemize}
\item Green vragen.
\end{itemize}

% section rondvraag (end)

\section{\underline{Sluiting}} % (fold)
\label{sec:sluiting}
\small{\emph{Logboeken!}}

% section sluiting (end)

\end{document}