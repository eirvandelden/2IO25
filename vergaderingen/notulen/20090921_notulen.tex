%
%  Notulen 21 September 2009
%  OGO 2.1 Groep 3
%
\documentclass[a4paper]{article}

% Packages!
\usepackage[utf8]{inputenc}
\usepackage[dutch]{babel}
\usepackage{hyperref}
\usepackage{amsmath}
\usepackage{amssymb}
\usepackage{boxedminipage}
\usepackage{listings}
\usepackage{ifpdf}

\ifpdf
\usepackage[pdftex]{graphicx}
\else
\usepackage{graphicx}
\fi

\title{Notulen Vergadering OGO 2.1 Groep 3 2009-2010}
\date{}

\begin{document}

\ifpdf
\DeclareGraphicsExtensions{.pdf, .jpg, .tif}
\else
\DeclareGraphicsExtensions{.eps, .jpg}
\fi

\section{\underline{Opening Vergadering}} % (fold)
\label{sec:opening_vergadering}
\begin{tabular}{ll}
  Datum:      & 21-09-2009\\
  Begin tijd: & 14:15\\
  Voorzitter: & Etienne van Delden\\
  Notulist:   & Jan\^ot Sijen\\
  & \\
  Aanwezig:   & Etienne van Delden\\
              & Tim Hermans\\
              & Tom van der Hoek\\
              & Jan\^ot Sijen\\
              & Robin Wolffensperger\\
              & Ron Vanderfeesten (tutor)\\
  Afwezig:    & \emptyset\\
\end{tabular}

% section opening_vergadering (end)

\section{\underline{Vaststelling Agenda}} % (fold)
\label{sec:vaststelling_van_de_agenda}

\begin{itemize}
\item Punten zijn goedgekeurd.
\end{itemize}

% section vaststelling_van_de_agenda (end)

%\section{\underline{Verslag vorige vergadering}} % (fold)
%\label{sec:bespreking_van_de_vorige_notulen}
%
%% section bespreking_van_de_vorige_notulen (end)

\section{\underline{Mededelingen}} % (fold)
\label{sec:mededelingen}

\begin{itemize}
\item Ron heeft de SVN uitgecheckt op zijn computer. Als iets in de SVN staat heeft Ron het ook.
\item .tex bestanden zijn geen probleem. (Ron houdt er zelfs van.)
\item Notulen zijn voor Ron leesbaar als ze voor ons leesbaar zijn. Veel tijd eraan besteden is niet vereist.
\end{itemize}

% section mededelingen (end)

\section{\underline{Correspondentie}} % (fold)
\label{sec:post_in_uit}

\begin{itemize}
\item -
\end{itemize}

% section post_in_uit (end)

\section{\underline{Feedback Werkplan}} % (fold)
\label{sec:feedback_werkplan}

\begin{itemize}
\item Heel volledig. Meest uitgebreide werkplan van iedereen!
\item Opmerking: Implementatie betekent test framework
\item We gaan dingen tegelijk doen in groepjes van 2 of 3. (Moet nog in werkplan vermeld worden.)
\item Eindverantwoordelijken waren wel toegekend, maar staan niet in het werkplan?
\item De printer eet blijkbaar wel eens pagina\'{}s op. Eindverantwoordelijken staan wel in het werkplan.
\item Closed algoritme v\`{o}\`{o}r alle anderen? Dit is omdat we eerst een algoritme gezamelijk doen, en daarna de anderen tegelijk.
\end{itemize}

% section feedback_werkplan (end)

\section{\underline{Voortgang OGO}} % (fold)
\label{sec:voortgang_ogo}
    
\begin{itemize}
\item Triangulations zijn gevoelig voor numerieke imprecisies, dus moeilijk te implementeren. Fouten opsporen hiermee kost veel tijd en moeite.
\item Er zijn simpelere algoritmes te vinden!
\end{itemize}

% section voortgang_ogo (end)

\section{\underline{Planning komende week}} % (fold)
\label{sec:planning_komende_week}

\begin{itemize}
\item Closed algorithm afmaken.
\item Preview op het concept-verslag doorsturen naar Ron?
\item Probeer zo vroeg mogelijk grip te krijgen op de running time.
\end{itemize}

% section planning_komende_week (end)

\section{\underline{Rondvraag}} % (fold)
\label{sec:rondvraag}

\begin{itemize}
\item Geen vragen.
\end{itemize}

% section rondvraag (end)

\section{\underline{Punten voor de volgende vergadering}} % (fold)
\label{sec:punten_voor_de_volgende_vergadering}

\begin{itemize}
\item Feedback op closed curve algoritme. (Indien af.)
\end{itemize}

% section punten_voor_de_volgende_vergadering (end)

\section{\underline{Sluiting}} % (fold)
\label{sec:sluiting}
\small{\emph{Logboek invullen!}}

% section sluiting (end)

\end{document}