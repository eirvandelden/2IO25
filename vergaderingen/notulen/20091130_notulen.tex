%
%  Notulen 30-11-2009
%  OGO 2.1 Groep 3
%
\documentclass[a4paper]{article}

% Packages!
\usepackage[utf8]{inputenc}
\usepackage[dutch]{babel}
\usepackage{hyperref}
\usepackage{amsmath}
\usepackage{amssymb}
\usepackage{boxedminipage}
\usepackage{listings}
\usepackage{ifpdf}

\ifpdf
\usepackage[pdftex]{graphicx}
\else
\usepackage{graphicx}
\fi

\title{Notulen Vergadering OGO 2.1 Groep 3 2009-2010}
\date{}

\begin{document}

\ifpdf
\DeclareGraphicsExtensions{.pdf, .jpg, .tif}
\else
\DeclareGraphicsExtensions{.eps, .jpg}
\fi

\section{\underline{Opening Vergadering}} % (fold)
\label{sec:opening_vergadering}
\begin{tabular}{ll}
  Datum:      & 30-11-2009\\
  Begin tijd: & 13:45\\
  Voorzitter: & Etienne van Delden\\
  Notulist:   & Jan\^ot Sijen\\
  & \\
  Aanwezig:   & Etienne van Delden\\
              & Tim Hermans\\
              & Tom van der Hoek\\
              & Jan\^ot Sijen\\
              & Robin Wolffensperger\\
              & Ron Vanderfeesten (tutor)\\
  Afwezig:    & $\emptyset$ \\
\end{tabular}

% section opening_vergadering (end)
\section{\underline{Vaststelling van de Agenda:}} % (fold)
\label{sec:vaststellingvandeagenda}

\begin{itemize}
\item Geen extra punten.
\end{itemize}

% section vaststellingvandeagenda (end)

\section{\underline{Verslag vorige vergadering:}} % (fold)
\label{sec:verslagvorigevergadering}

\begin{itemize}
\item Waarom is Ron scheel? Misschien beter weghalen.
\end{itemize}

% section verslagvorigevergadering (end)

\section{\underline{Mededelingen:}} % (fold)
\label{sec:mededelingen}

\begin{itemize}
\item Peach is actief. (Vak OGO.)
\item Op Peach kun je je programma inzenden en testen.
\item Advies: Lever zo snel mogelijk iets in, en zorg dat je zo snel mogelijk een programma hebt dat Peach slikt.
\end{itemize}

% section mededelingen (end)

\section{\underline{Feedback concept verslag:}} % (fold)
\label{sec:feedbackconceptverslag}

\begin{itemize}
\item Concept verslag is nog niet doorgelezen. (Stond niet in Ron's zakcomputer ding.)
\item Ron leest het nog door voor woensdag.
\end{itemize}

% section feedbackconceptverslag (end)

\section{\underline{Status update:}} % (fold)
\label{sec:statusupdate}

\begin{itemize}
\item Eerste iteratie nieuwe oplossing is compleet. (for-loop algoritme)
\item Tweede iteratie is bijna af.
\item Ron is onder de indruk. We zijn in de afgelopen weken ver gekomen!
\item De stap naar open/intersect zal niet moeilijk zijn. (Maak kleine wijzigingen hiervoor.)
\item Maandagavond moet de code worden ingeleverd.
\item Ron probeert ons wat extra dagen de tijd te geven, maar hiervoor is geen garantie.
\item Focus op de code, presentatie is niet echt belangrijk.
\end{itemize}

% section statusupdate (end)

\section{\underline{Planning vandaag:}} % (fold)
\label{sec:planningvandaag}

\begin{itemize}
\item Tweede iteratie, puntjes op de i zetten.
\item Nieuwe aanpassing algoritme afmaken.
\end{itemize}

% section planningvandaag (end)

\section{\underline{Rondvraag:}} % (fold)
\label{sec:rondvraag}

\begin{itemize}
\item Geen vragen.
\end{itemize}

% section rondvraag (end)

\section{\underline{Sluiting}} % (fold)
\label{sec:sluiting}
\small{\emph{Logboeken!}}

% section sluiting (end)

\end{document}