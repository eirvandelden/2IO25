%
%  Notulen 23-11-2009
%  OGO 2.1 Groep 3
%
\documentclass[a4paper]{article}

% Packages!
\usepackage[utf8]{inputenc}
\usepackage[dutch]{babel}
\usepackage{hyperref}
\usepackage{amsmath}
\usepackage{amssymb}
\usepackage{boxedminipage}
\usepackage{listings}
\usepackage{ifpdf}

\ifpdf
\usepackage[pdftex]{graphicx}
\else
\usepackage{graphicx}
\fi

\title{Notulen Vergadering OGO 2.1 Groep 3 2009-2010}
\date{}

\begin{document}

\ifpdf
\DeclareGraphicsExtensions{.pdf, .jpg, .tif}
\else
\DeclareGraphicsExtensions{.eps, .jpg}
\fi

\section{\underline{Opening Vergadering}} % (fold)
\label{sec:opening_vergadering}
\begin{tabular}{ll}
  Datum:      & 23-11-2009\\
  Begin tijd: & 13:45\\
  Voorzitter: & Etienne van Delden\\
  Notulist:   & Jan\^ot Sijen\\
  & \\
  Aanwezig:   & Etienne van Delden\\
              & Tim Hermans\\
              & Tom van der Hoek\\
              & Jan\^ot Sijen\\
              & Robin Wolffensperger\\
              & Ron Vanderfeesten (tutor)\\
  Afwezig:    & $\emptyset$ \\
\end{tabular}

% section opening_vergadering (end)
\section{\underline{Mededelingen:}} % (fold)
\label{sec:mededelingen}

\begin{itemize}
\item Er komt ook nog een presentatie aan het einde. Deze moet in het Engels gehouden worden.
\item Niet iedereen hoeft een presentatie te houden, maar iedereen moet er een voorbereiden. Pas op het laatste moment wordt bekend gemaakt wie de presentaties moet doen.
\end{itemize}

% section mededelingen (end)

\section{\underline{Feedback:}} % (fold)
\label{sec:feedback}

\begin{itemize}
\item Minder goed nieuws.
\item Ron is scheel, maar dat maakt niet uit.
\item Het is supergoed dat we zoveel werk hebben gedaan in maar een week tijd.
\item De tijd is een beetje het probleem. Over 2 weken moet de code al ingeleverd zijn.
\item Algemeen: we zeggen veel, maar we gebruiken weinig definities.
\item Na de introduction, eigenlijk: "Definitions", dan de uitleg van het algoritme.
\item Wees zo precies mogelijk als je kan zijn. (Zo min mogelijk woorden zo veel mogelijk zeggen zonder dat het onduidelijk wordt.)
\item Neem als voorbeeld definitions in een wiskundeboek. (Stelling, proof, etc.)
\item Gebruikt definities ook in de beschrijving van het algoritme.
\item Volgens de docent heeft maar één iemand het verslag geschreven, het leek alsof het niet geproofread was. (Klopt helemaal niet!)
\item De pseudocode was wel goed.
\item Draai experiments!
\item Deden we nu maar stoelmaken. Dat is wel makkelijk.
\end{itemize}

% section feedback (end)

\section{\underline{Backtracking:}} % (fold)
\label{sec:backtracking}

\begin{itemize}
\item We moeten backtracking vergeten.
\item Helemaal teruggaan naar een simpel algoritme, paar test cases runnen. Dat wordt het eerste deel van het verslag.
\item vb: "We proberen NN, zo werkt het, dit zijn de resultaten, etc"
\item Verbeteringen toepassen, repeat.
\end{itemize}

% section backtracking (end)

\section{\underline{Ongecategoriseerd:}} % (fold)
\label{sec:ongecategoriseerd}

\begin{itemize}
\item Het is belangrijker om geen intersects te hebben dan een ander punt ergens kiezen!
\item Vergelijking tussen algoritmes mogen op het einde.
\item Running time analysis: wees precies!
\item Let op multicurve.
\end{itemize}

% section ongecategoriseerd (end)

\section{\underline{To-do:}} % (fold)
\label{sec:todo}

\begin{itemize}
\item Janôt: Maak for-loop zoals beschreven in projectwijzer. Input/output. Voor vandaag!
\item Tim: Experimenten. Running time (gebruik ook definities)! Vandaag/woensdag.
\item Robin: Nearest neighbor beschrijven met definities. Vandaag.
\item Tom: Functionele beschrijving, "n00b explanation voor hbo programmeurs". (Leg idee uit. Vergeet datastructuren niet! vb "A is een array die dit representeert.") Vandaag.
\item Etienne: Theoretische running time van het algoritme. (Die is vaak anders dan de experimentele running time. Ook weer definities gebruiken.)
\item Conclusie, vergelijking van running times.
\item Conclusie van conclusies. "Eerste deed het goed, maar niet voor ... Tweede was beter, maar heeft nog steeds ..."
\end{itemize}

% section todo (end)

\section{\underline{Sluiting}} % (fold)
\label{sec:sluiting}
\small{\emph{Logboeken!}}

% section sluiting (end)

\end{document}