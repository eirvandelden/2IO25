%
%  Notulen 07-12-2009
%  OGO 2.1 Groep 3
%
\documentclass[a4paper]{article}

% Packages!
\usepackage[utf8]{inputenc}
\usepackage[dutch]{babel}
\usepackage{hyperref}
\usepackage{amsmath}
\usepackage{amssymb}
\usepackage{boxedminipage}
\usepackage{listings}
\usepackage{ifpdf}

\ifpdf
\usepackage[pdftex]{graphicx}
\else
\usepackage{graphicx}
\fi

\title{Notulen Vergadering OGO 2.1 Groep 3 2009-2010}
\date{}

\begin{document}

\ifpdf
\DeclareGraphicsExtensions{.pdf, .jpg, .tif}
\else
\DeclareGraphicsExtensions{.eps, .jpg}
\fi

\section{\underline{Opening Vergadering}} % (fold)
\label{sec:opening_vergadering}
\begin{tabular}{ll}
  Datum:      & 07-12-2009\\
  Begin tijd: & 14:00\\
  Voorzitter: & Etienne van Delden\\
  Notulist:   & Jan\^ot Sijen\\
  & \\
  Aanwezig:   & Etienne van Delden\\
              & Tim Hermans\\
              & Tom van der Hoek\\
              & Jan\^ot Sijen\\
              & Robin Wolffensperger\\
              & Ron Vanderfeesten (tutor)\\
  Afwezig:    & $\emptyset$ \\
\end{tabular}

% section opening_vergadering (end)
\section{\underline{Vaststelling Agenda:}} % (fold)
\label{sec:vaststellingagenda}

\begin{itemize}
\item Ron heeft het verslag doorgekeken. (Zie Mededelingen)
\end{itemize}

% section vaststellingagenda (end)

\section{\underline{Verslag vorige vergadering:}} % (fold)
\label{sec:verslagvorigevergadering}

\begin{itemize}
\item Geen opmerkingen.
\end{itemize}

% section verslagvorigevergadering (end)

\section{\underline{Mededelingen:}} % (fold)
\label{sec:mededelingen}

\begin{itemize}
\item Goed en slecht nieuws.
\item Eerst, slecht nieuws: geen deadline extension.
\item Goed nieuws: Het verslag is goed!
\item Het kan nóg beter, maar we zijn netjes te werk gegaan.
\item Formeel gedeelte is goed, maar af en toe kan er meer uitleg bij.
\item Ron\'s commentaar is onleesbaar.
\end{itemize}

% section mededelingen (end)

\section{\underline{Status update OGO:}} % (fold)
\label{sec:statusupdateogo}

\begin{itemize}
\item Closed curve is af: improved nearest neighbor.
\item Open curve: nieuwe functie.
\item Intersection: directed nearest neighbor (heeft nog werk nodig)
\item Up to five is af: nearest neighbor met aanpassing voor meerdere curves. (Trein-code)
\item Console applicatie is af.
\end{itemize}

% section statusupdateogo (end)

\section{\underline{Planning vandaag:}} % (fold)
\label{sec:planningvandaag}

\begin{itemize}
\item Code krijgt voorrang.
\item Iemand mag over de schouders heen kijken. (Om programmeerfouten te voorkomen.)
\end{itemize}

% section planningvandaag (end)

\section{\underline{Planning de rest van de week:}} % (fold)
\label{sec:planningderestvandeweek}

\begin{itemize}
\item Verslag moet vrijdag af. (11 December)
\item Woensdag zijn presentaties, zal ongeveer 2 uur duren. (3 groepen presenteren, 20 minuten per groep)
\item Er moet nog een presentatie gemaakt worden.
\item Ron heeft liever een slechte presentatie en goed verslag dan andersom.
\item Etienne vindt presentaties maken leuk.
\item Robin staat er niet om te springen.
\item Presentatie is ons \"verkooppraatje\". (Het is amazing! Waarom is het amazing? Hoe werkt het?)
\item Volgende maandag eindbespreking? (Niet zeker.)
\end{itemize}

% section planningderestvandeweek (end)

\section{\underline{Rondvraag:}} % (fold)
\label{sec:rondvraag}

\begin{itemize}
\item Geen vragen.
\end{itemize}

% section rondvraag (end)

\section{\underline{Punten voor volgende vergadering:}} % (fold)
\label{sec:puntenvoorvolgendevergadering}

\begin{itemize}
\item Error 201 SIGSEGV: Er is geen volgende vergadering.
\end{itemize}

% section puntenvoorvolgendevergadering (end)

\section{\underline{Sluiting}} % (fold)
\label{sec:sluiting}
\small{\emph{Logboeken!}}

% section sluiting (end)

\end{document}