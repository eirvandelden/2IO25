%
%  Notulen 5 October 2009
%  OGO 2.1 Groep 3
%
\documentclass[a4paper]{article}

% Packages!
\usepackage[utf8]{inputenc}
\usepackage[dutch]{babel}
\usepackage{hyperref}
\usepackage{amsmath}
\usepackage{amssymb}
\usepackage{boxedminipage}
\usepackage{listings}
\usepackage{ifpdf}

\ifpdf
\usepackage[pdftex]{graphicx}
\else
\usepackage{graphicx}
\fi

\title{Notulen Vergadering OGO 2.1 Groep 3 2009-2010}
\date{}

\begin{document}

\ifpdf
\DeclareGraphicsExtensions{.pdf, .jpg, .tif}
\else
\DeclareGraphicsExtensions{.eps, .jpg}
\fi

\section{\underline{Opening Vergadering}} % (fold)
\label{sec:opening_vergadering}
\begin{tabular}{ll}
  Datum:      & 05-10-2009\\
  Begin tijd: & 14:15\\
  Voorzitter: & Etienne van Delden\\
  Notulist:   & Jan\^ot Sijen\\
  & \\
  Aanwezig:   & Etienne van Delden\\
              & Tim Hermans\\
              & Tom van der Hoek\\
              & Jan\^ot Sijen\\
              & Robin Wolffensperger\\
              & Ron Vanderfeesten (tutor)\\
  Afwezig:    & \emptyset\\
\end{tabular}

% section opening_vergadering (end)

\section{\underline{Vaststelling Agenda}} % (fold)
\label{sec:vaststelling_van_de_agenda}

\begin{itemize}
\item Ron wil graag de onveranderde driver nog een keer zien aan het einde.
\end{itemize}

% section bespreking_van_de_vorige_notulen (end)

\section{\underline{Mededelingen}} % (fold)
\label{sec:mededelingen}

\begin{itemize}
\item Er komen toch testcases! Binnen twee weken. (Kan ook 4 weken duren met die mensen.) (Zet dit niet in de notulen.)
\item Testcases zijn de meest basic dingen. Moeilijke dingen zullen er niet in zitten. We moeten dus zelf ook test cases maken!
\item Eigen framework is goed! Er is blijkbaar ook een framework voor iedereen beschikbaar gemaakt.
\end{itemize}

% section mededelingen (end)

\section{\underline{Notulen vorige keer}} % (fold)
\label{sec:notulen_vorige_keer}

\begin{itemize}
\item Zagen er goed uit. Niet te lang, niet te kort.
\item Genoege informatiedichtheid.
\end{itemize}

% section notulen_vorige_keer (end)

\section{\underline{Correspondentie}} % (fold)
\label{sec:post_in_uit}

\begin{itemize}
\item Geen post.
\end{itemize}

% section post_in_uit (end)

\section{\underline{Algoritmes}} % (fold)
\label{sec:algoritmes}

\begin{itemize}
\item We zijn nog bezig met het eerste algoritme.
\item Daar gaan we gewoon mee door!
\end{itemize}

% section algoritmes (end)

\section{\underline{Voortgang algemeen}} % (fold)
\label{sec:voortgang_ogo}
    
\begin{itemize}
\item Vraag: Hoe moeten we punten langsgaan? \\
  In de tekst staat tegen de klok in, maar in het voorbeeld gaan ze met de klok mee. \\
  Ron gaat het navragen!
\item Het is misschien verstanding als een aantal mensen al naar het volgende algoritme gaat kijken. \\
  (Tim wil dat best wel doen.)
\end{itemize}

% section voortgang_ogo (end)

\section{\underline{Planning komende week}} % (fold)
\label{sec:planning_komende_week}

\begin{itemize}
\item Eind deze week een concept voor het verlsag inleveren. \\
  Oh nee, dat duurt nog een maand.
\item We moeten voor de echte deadline wel al wat hebben, maar echte conclusies hoeven nog niet.
\end{itemize}

% section planning_komende_week (end)

\section{\underline{Rondvraag}} % (fold)
\label{sec:rondvraag}

\begin{itemize}
\item -
\end{itemize}

% section rondvraag (end)

\section{\underline{Sluiting}} % (fold)
\label{sec:sluiting}
\small{\emph{Logboek invullen!}}

% section sluiting (end)

\end{document}