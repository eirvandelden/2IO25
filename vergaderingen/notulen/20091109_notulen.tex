%
%  Notulen 19 Oktober 2009
%  OGO 2.1 Groep 3
%
\documentclass[a4paper]{article}

% Packages!
\usepackage[utf8]{inputenc}
\usepackage[dutch]{babel}
\usepackage{hyperref}
\usepackage{amsmath}
\usepackage{amssymb}
\usepackage{boxedminipage}
\usepackage{listings}
\usepackage{ifpdf}

\ifpdf
\usepackage[pdftex]{graphicx}
\else
\usepackage{graphicx}
\fi

\title{Notulen Vergadering OGO 2.1 Groep 3 2009-2010}
\date{}

\begin{document}

\ifpdf
\DeclareGraphicsExtensions{.pdf, .jpg, .tif}
\else
\DeclareGraphicsExtensions{.eps, .jpg}
\fi

\section{\underline{Opening Vergadering}} % (fold)
\label{sec:opening_vergadering}
\begin{tabular}{ll}
  Datum:      & 09-11-2009\\
  Begin tijd: & 13:50\\
  Voorzitter: & Etienne van Delden\\
  Notulist:   & Jan\^ot Sijen\\
  & \\
  Aanwezig:   & Etienne van Delden\\
              & Tim Hermans\\
              & Tom van der Hoek\\
              & Jan\^ot Sijen\\
              & Robin Wolffensperger\\
              & Ron Vanderfeesten (tutor)\\
  Afwezig:    & $\emptyset$ \\
\end{tabular}

% section opening_vergadering (end)

\section{\underline{Vaststelling Agenda}} % (fold)
\label{sec:vaststelling_van_de_agenda}

\begin{itemize}
\item Hebben we iets gedaan tijdens de tentamenweek? Ja, dat hebben we.
\end{itemize}

% section vaststelling_van_de_agenda (end)

\section{\underline{Verslag vorige vergadering}} % (fold)
\label{sec:bespreking_van_de_vorige_notulen}

\begin{itemize}
\item Opmerkingen op de notulen? Ron heeft ze niet bekeken! Ze staan nog niet op de SVN! (EDIT: Nu wel.)
\end{itemize}

% section bespreking_van_de_vorige_notulen (end)

\section{\underline{Feedback conceptverslag}} % (fold)
\label{sec:mededelingen}

\begin{itemize}
\item Conceptverslag valt een beetje tegen, we werken wel hard!
\item 3 pagina's pseudocode en... niks.
\item We hoeven niet alleen te schrijven over wat wel werkt, we kunnen ook foute implementaties opschrijven en waarom het fout gaat!
\item ``Uitleg is belangrijker dan pseudocode''
\item Uitleggen hoe een KD-tree werkt hoeft niet, verwijs naar literatuur, geef alleen criterium van tree. (Wat links, wat rechts.)
\item Je moet je algoritme verkopen.
\item Bij augmentation definieer je niet de basis datastructuur opnieuw. Noteer alleen maar de augments.
\item Geldt ook voor andere basisdingen. (Het concept backtracking hoeft dus ook niet.)
\item Pseudocode is een moker? Gebruik liever een scalpel.
\item We beginnen de introductie met ``in section blaat doen we blaat'', maar dat hoeft niet in een paper. Dat gaat gewoon in de inhoudsopgave.\\
  (Bijv. ``We're going to build a KD-tree with augmentation''\\
   niet: ``We're going to build a KD-tree with augmentation in section 14'')
\item We leveren goed werk, maar het document kan beter!
\item Je moet het aan een HBO/universitair programmeur kunnen geven, en die moet ermee om kunnen gaan.
\item We hebben de tijd.
\end{itemize}

% section update_algorithms (end)

\section{\underline{Voortgang OGO}} % (fold)
\label{sec:voortgang_ogo}
    
\begin{itemize}
\item Programma doet het best goed! Alleen moeten intersections nog verholpen worden.
\end{itemize}

% section voortgang_ogo (end)

\section{\underline{Rondvraag}} % (fold)
\label{sec:rondvraag}

\begin{itemize}
\item Feedback peer reviews? Donderdag is de volgende meeting.
\end{itemize}

% section rondvraag (end)

\section{\underline{Punten voor de volgende vergadering}} % (fold)
\label{sec:punten_voor_de_volgende_vergadering}

\begin{itemize}
\item Nieuwe versie conceptverslag
\end{itemize}

% section punten_voor_de_volgende_vergadering (end)

\section{\underline{Sluiting}} % (fold)
\label{sec:sluiting}
\small{\emph{Dat was het dan weer!}}

% section sluiting (end)

\end{document}